\documentclass{article}

\usepackage[english]{babel}
\usepackage{graphicx}
\usepackage[a4paper]{geometry}
%%\usepackage[a4paper,top=2cm,bottom=2cm,left=3cm,right=3cm,marginparwidth=1.75cm]{geometry}
\begin{document}
\begin{titlepage}
\begin{center}

        \includegraphics[width=0.4\textwidth]{NSU_logo}
        
        \LARGE
      	  \textbf{North South University} \\
        Department of Electrical and Computer Engineering\\		
                \vspace*{0.2cm}
	 \textbf{Project Proposal}\\
        
        \vspace*{0.5 cm}
        
        \Huge
        \textbf{``E-commerce Website for Clothing Brands'' }
      
      \vspace{0.5cm}
        \Large
        \textrm{Database System Lab}\\
        \textrm{CSE311L}\\
        \textrm{Section: 5}\\
        \textrm{Group: 11}\\
        \textrm{Semester: Summer 2021}\\
        \textrm{Course Faculty: Nadeem Ahmed, Senior Lecturer}\\
        \textrm{Lab Instructor: Nazmul Alam Dipto}\\



        
        \vspace{1cm}
        \Large
       submitted by\\
        \textbf{Tanvir Khan}
       \texttt{1911481642}  
        
        \textbf{Tanvir Ibne Hossain}
        \texttt{1912205042}     
        
        \textbf{Nahedul Bar Chowdhury Ruhul}
        \texttt{1912656642}  
        

    \end{center}

\end{titlepage}




\section{Introduction}

The biggest e-commerce sites in Bangladesh e.g., Daraz, Rokomari, Pickaboo, Evaly etc. all focuses their business either as all-in-one solution for all products or some generic popular streams like books or electronics. But one side we see less of is e-commerce sites focusing on clothing. Through this project our web application will try to fill this void. Our web application will be a one stop solution for all the clothing and apparel needs for people from all spere of life. This web application will contain all product information with picture for the customers to judge. They can add product to the cart/Wishlist to order later or can order a product immediately. There will a merchant side of the application from where merchants can add products to the site and see how a particular product is performing. In Bangladesh almost every fashion brand is Dhaka or urban centric. With our web app they can reach people throughout the country. A online e-commerce site can be becon to a revolutionary change on how people shops clothes.



\section{Objective}

\begin{enumerate}
\item Easy to buy fashion product from home.
\item Give the customers more option even if it is in rural area.
\item Can access clothing items from all over the country.
\item Extend the business with low cost. 
\item The supply chain is more flexible for merchant.
\item Customers get the product comparatively at a better price.
\end{enumerate}



\section{Target Customers}
As this is an e-commerce site so our target customers are people from all sphere of society. But we should focus on the young adults more as most of the internet users are in this bunch. 
Meaning our target would be:
\begin{itemize}
\item Male
\item Female
\item Kids

\end{itemize}




\section{Value Proposition}

\subsection{Eliminates Inventory Cost}
One of the biggest drawbacks of running an offline retail business is it needs a lot of inventory cost. But in e-commerce the merchant can deliver their product from their warehouse. Thus, inventory cost eliminates. 

\subsection{Lower Costs}
In e-commerce the merchant can sell the product with less employee and less place. Which obviously reduce cost maximize the profit and the gets enough room to give offer to their customers so that they will buy more. 

\subsection{Flexibility}
One customer can buy goods from anywhere they want. The customers no need to go out and find some times for shopping. They can order and get delivery to their home.


\section{Web Application Feature and description}
Customers have to create an account to use most of the features. After creating an account, a customer can: 
\begin{itemize}
\item Place order.
\item Add products in Wishlist/cart.
\item Will get Recommended Product suggestions. 
\item Users Can Rate and Review Shops/Products.
\item Access history of their past purchases.
\item View notifications for sale, Product Arrival.
\item Can use advance Payment System for checkout.
\end{itemize}

        \vspace*{0.5 cm}
Merchants are essential as they will bring the goods. So there will be login section for them to. From there they can monitor aspects of the bussiness.  After registering a merchant can:

\begin{itemize}
\item Add New Products 
\item Change Prices
\item Get notification On Order 
\item See Inventory Information
\item Get ratings of the products
\item See reviews of the products
\end{itemize}


\pagebreak 

\section{Tools and Resources}
\begin{itemize}
\item HTML
\item JavaScript
\item MySQL
\item PHP
\item Web server
\item  Different APIs
\end{itemize}


\section{Challenge}
One of the biggest challenges for this idea to work is the consumer behavior of the people. People are a bit skeptical in buying clothes from online sites. So, ensuring top notch customer services is a must. The web application needs to be user friendly so that people of ages and from every aspect of life can navigate their way through to find new items to buy. There will be a huge number of products and it will be a challenge to show appropriate products to the customers. Then as an e-commerce site it will be a challenge to give financial security to customers. Finally, sending the ordered products to the doorsteps of the people will also be a huge undertake.


\end{document}